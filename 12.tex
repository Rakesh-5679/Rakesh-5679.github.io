documentclass[conference]{IEEEtran} 
\IEEEoverridecommandlockouts 
\usepackage{cite} 
\usepackage{url} 
\usepackage{amsmath,amssymb,amsfonts} 
\usepackage{algorithmic} 
\usepackage{graphicx} 
\usepackage{textcomp} 
\usepackage{xcolor} 
\usepackage{stfloats} 
\usepackage{multicol} 
\def\BibTeX{{\rm B\kern-.05em{\sc i\kern-.025em b}\kern-.08em 
T\kern-.1667em\lower.7ex\hbox{E}\kern-.125emX}} 
\begin{document} 
\title{COVID-19 and Comorbidities Pandemic Effects: A Review \\} 
\author{\IEEEauthorblockN{Mamatha.L} 
\IEEEauthorblockA{ Department of Computer Science and Engineering, SIET,TUMKUR}} 
\maketitle 
\begin{abstract} 
%COVID-19 is a contagious disease caused by a virus,the Severe Acute Repository Syndrome coronavirus -2 
(SARS-CoV-2). The first known case was identified in Wuhan, China, in December 2019.The co-existence of 
two or more illness or disorders in a single patient. This study received the Comorbidities and vulnerability 
among people pre and post COVID-19. An advanced database search is made related to modern technology 
towards COVID-19, In recent year, many health cares have increased and attracted attention, to make the issue 
serious by limited data healthy many techniques like Machine Learning and modern computational intelligence 
are applied to provide the disease. 
A virus known as Severe Acute Repository Syndrome Coronavirus-2 (SARS-CoV-2) is the source of the 
infectious disease COVID-19. The first case of COVID – 19 is recorded in Wuhan, China, in December 2019. If 
a person had COVID-19 and another disease or condition, then COVID-19 and other diseases would be 
considered comorbidities. Researchers began studying the link between people with certain comorbidities and 
the COVID-19 disease after the start of the pandemic. This paper gives reviews on the effects of COVID-19 and 
its comorbidities. Machine learning and Deep learning techniques are used to detect the severity of COVID-19 
and the comorbidities from the existing literature work. 
\end{abstract} 
\begin{IEEEkeywords} 
COVID-19, Comorbidity, Deep Learning,Machine Learning, SARS-CoV-2. 
\end{IEEEkeywords} 
\section{Introduction} 
\IEEEPARstart{T}he disease known as COVID-19 is mostly caused by the SARS-CoV-2 virus, which was 
discovered in China in December 2019. As a result, it spread fast around the world. 26 
COVID-19 spreads through air-borne patients containing the virus or the droplets. The Transmission can occur 
through the contaminated fluids in the eyes, nose or mouth, and, rarely, via contaminated surfaces. 
COVID-19 affects different people in different ways. Most infected people will develop mild to moderate 
illness and recover without hospitalization. The symptoms of this disease can be categorized into three types 
such as most common, less common, and serious problems.. The Serious symptoms are difficulty in breathing 
or shortness of breath, loss of speech or mobility, or confusion and chest pain. Fig. 1 shows the three categories 
of symptoms of COVID-19 disease. 
The Covid-19 patient seeks immediate medical attention if he has serious symptoms. People with mild 
symptoms who are otherwise healthy should manage their symptoms at home. On average it takes 5–6 days 
from when someone is infected with the virus for symptoms to show, however it can take up to 14 days. 
This disease weakens the body's functions, making it more challenging for the body to fight off invaders like 
viruses and bacteria and to get rid of the disease's main cause. It may be extremely stressful on the body to have 
two or more diseases present at once, and the affected individual may take longer to recover than someone 
without comorbidity. 
COVID-19 spreads through, air-borne patients containing the virus or the droplets. Transmission can occur 
through the contaminated surface. 
The Transmission can occur through the contaminated fluids in the eyes, nose or mouth, and, rarely, via 
contaminated surfaces. COVID-19 testing through swab as well as rRT-PCR and CT scan methods. 
The Table 1 is the COVID cases (43452164)and Death cases (525116) in India (March 2020 to October 2022). 
\begin{table*} 
\caption{COVID cases in States / Union Territories} 
\label{tab:t1} 
\centering 
\begin{tabular} 
{p{0.10\linewidth} p{0.10\linewidth}p{0.10\linewidth}p{0.10\linewidth}p{0.10\linewidth}p{0.10\linewidth} 
p{0.10\linewidth}p{0.10\linewidth}p{0.10\linewidth}p{0.10\linewidth}p{0.10\linewidth} p{0.10\linewidth} 
\hline 
\textbf{Sl.No.} & \textbf{COVID Cases} & \textbf{Death Cases} & \textbf{Result}\\ 
\hline 
Yamaç \emph{et al.} \cite{yamac2021convolutional} 2021 & 
SVM, KNN, and CRC & X-ray images of QaTa-Cov19 & Sensitivity: 98\% 
\newline 
Specificity:95\% 
\end{tabular} 
\end{table*} 
\begin{figure}[htp] 
\centering 
\includegraphics[width=8cm,height=9 
cm]{Sym.JPG} \caption{Symptoms of COVID-19 Effects} 
\label{fig:f1} 
\end{figure} 27 
\subsection*{A. COMORBIDITIES} 
Comorbidities are the presence of two or more diseases in the same person. In the case of COVID-19, if a 
person had COVID-19 and another disease, then COVID-19 and the other condition would be comorbidities. 
So, for example, if a person had diabetes and then developed COVID-19, they would have two diseases—or 
comorbidities. 
\newline 
Table 2 is existing diagnosis algorithm for comorbidities are incorporated. 
\begin{figure*}[htp] 
\centering 
\includegraphics[width=14cm,height=8cm]{Suresh_2.png} 
\caption{Death Rate by Comorbidities diseases} 
\label{fig:f2} 
\end{figure*} 
\subsection*{B. DIAGNOSIS ALGORITHMS} 
Table 3 is existing diagnosis algorithm for diseases are incorporated. 
\begin{table*} 
\caption{Existing diagnosis algorithm for comorbidities} 
\label{tab:t1} 
\centering 
\begin{tabular} 
{p{0.20\linewidth} p{0.15\linewidth}p{0.20\linewidth}p{0.20\linewidth}p{0.20\linewidth}} 
\hline 
\textbf{Author [Reference] Year} & \textbf{Model / Classifier} & \textbf{Dataset} & \textbf{Result}\\ 
%\textbf{Age in Years} & 
\hline 
Yamaç \emph{et al.} \cite{yamac2021convolutional} 2021 & 
SVM, KNN, and CRC & X-ray images of QaTa-Cov19 & Sensitivity: 98\% 
\newline 
Specificity:95\% 
\\\\ 
Wanyan \emph{et al.} \cite {wanyan2020relational} 2021 & DL/ML - Heterogeneous Graph model and LSTM 
& Electronic Health Records & Training set Accuracy:0.935 
\newline 
Test set Accuracy:0.847 
\\\\ 
Liu \emph{et al.} \cite {liu2020risk} 2020 & Risk factors of infection in COVID-19: RT-PCR & 11580 of 
persons & RR: 5.29, 95\%CI: 3.76-7.46 
\\\\ 
Gammulle \emph{et al.} \cite {gammulle2021multi} 2021 & 
DL-SE-ResNet-50 & 307 patients of radiomics & Accuracy:86.63\% 
\\\\ 
Pathak \emph{et al.} \cite {pathak2020deep} 2020 & AI- CNN, ANN & 2842 CT Scan images & 
Accuracy:1.79, AUC:1.53 
\newline f-measure:1.84, Sensitivity:1.93 
\newline 
Specificity:0.44, recall:1.64 
\newline 
Precision:1.53 
\\\\ 
Han \emph{et al.} \cite {han2020accurate} 2020 & 28 
DL/ML - CatBoost / SVM / NB / LR & EMRs: Test 1 (753) 
Test 2(2123) 
& Accuracy:97.6, AUC:99 
\newline f1 score:97.9,Precision:97.9 
\newline 
Cohen Kappa:95.7, recall:97.9 
\\\\ 
\hline 
\end{tabular} 
\end{table*} 
\begin{table*} 
\caption{Existing diagnosis algorithm for Diseases} 
\label{tab:t2} 
\centering 
\begin{tabular} 
{p{0.20\linewidth} p{0.15\linewidth}p{0.20\linewidth}p{0.20\linewidth}p{0.20\linewidth}} 
\hline 
\textbf{Author [Reference] Year} & \textbf{Model / Classifier} & \textbf{Dataset} & \textbf{Result}\\ 
%\textbf{Age in Years} & 
\hline 
Mehrabadi \emph{et al.} \cite{mehrabadi2021detection} 2021& 
DL – CNN / LSTM & 77972 samples of Age & ACU: 0.81 
\\\\ 
Singh \emph{et al.} \cite {singh2021ediapredict} 2021 & ML - XGBoost, Decision Tree, Random Forest, NN 
& PIMA Indian Diabetes Data 9 attributes & Accuracy:92.21\% 
\\\\ 
Karboub \emph{et al.} \cite {karboub2020automated} 2022 & AI-CNN, SVM, KNN, NB & 72000 ECG Beats 
& Accuracy:99.2\% 
\\\\ 
Handy \emph{et al.} \cite {Handy923} 2022 & 
logistic and Cox Regreession & 972971 Female and CHA samples & Odds Ratio:95\% 
\\\\ 
Aburuz \emph{et al.} \cite {aburuz2022clinical} 2022 & logistic Regression Analysis & 3296 patients Age & 
Odds Ratio:2.87 
\\\\ 
Yu \emph{et al.} \cite {yu2020role} 2021 & 
AI - CatBoost / SVM / NB / LR & EMRs: Test 1 (753) 
\newline 
Test 2(2123) 
& T1-Accuracy: 84.7\% 
\newline 
T2-Accuracy:96.7\% 
\\\\ 
\hline 
\end{tabular} 
\end{table*} 29 
\section{ Literature Review } 
In literature reviews severity of COVID-19 and comorbid effects and diagnosis algorithms are reviewed and 
explained. 
\newline 
Devarajan \emph{et al.} \cite {devarajan2021healthcare} investigated on the effect of the COVID-19 flare-up 
on medical services tasks and creates AI based anticipating models utilizing time series information to predict 
the movement of COVID-19 and further utilizing prescient examination to all the more likely oversee medical 
care activities. 
Gu \emph{et al.} \cite {gu2021epidemic} proposed novel plague risk evaluation technique in light of the 
granular information gathered by correspondence stations. The calculation scourge hazard of the 
correspondence stations in various stretches by consolidating the quantity of tainted people and the manner in 
which they go through the stations has been examined. 
Li \emph{et al.} \cite {li2021rapid} the component determination strategy in view of stepwise relapse is utilized 
to handle the COVID-19 pandemic informational index from January 13,2020 to January 16, 2021 in the United 
States. After measurable testing, the Auto Regressive Integrated Moving Average (ARIMA) model and the 
better ARIMAX model in view of element determination rapidly settles the improvement pattern of the 
COVID-19 pestilence in the US. 
Elahraf \emph{et al.} \cite {elahraf2021service} proposed a help situated structure that permits dynamic piece 
and the executives of such quiet consideration plans expecting a fitting information base and accessibility of 
web administrations points of interaction of the fundamental frameworks of guardians and specialist 
organizations. 
Feil-Seifer \emph{et al.} \cite {feil2020next} propose Human-Robot Interaction Research (HRI) research acted 
before long will be changed in essentially various ways; the powerlessness to perform or expect the future 
presentation of in-person human subjects research, particularly research including material or multiparty 
cooperation, will change both the prevailing systemic methods utilized by HRI specialists and the very research 
questions that the field decides to and can address. 
Khan \emph{et al.} \cite {khan2020prediction} foresee whether the cesarean segment is important with the 
assistance of information mining and subsequently, expanding the security of the mother and infant during and 
after labor by keeping away from superfluous cesarean area has been tended to. To accomplish the goal, three 
unique gathering forecast models in view of XGBoost, AdaBoost and Catboost have been created. 
Jokinen \emph{et al.} \cite {jokinen2021detection} utilized atomic elements reproductions in combined solvents as 
one with virtual screening to distinguish little particles that could be possible inhibitors of S protein in turn ACE2 
collaboration. Likewise, an original conformity of the S protein was found that could be balanced out by little 
particles to restrain connection to ACE2. 
Kyono \emph{et al.} \cite {kyono2021exploiting} propose a model determination strategy for utilizing model 
expectations on an objective space without names by taking advantage of the area invariance of causal design. We 
expect or gain a causal chart from the source space and select models that produce anticipated disseminations in the 
objective space that have the most elevated probability of accommodating our causal diagram. 30 
\section{Conclusions} 
Our meta- analysis showed that diabetes increases the mortality of patients with COVID-19. Even since COVIS-19 
had occur in environment may her infected by SAR-COV-2. The rate of infection is detected by virus health data and 
cough detectors and even by long health detectors.. Many biomarkers are used to identify the condition of the 
patients. In this paper, COVID-19 was detected by different methods like Deep Learning Network Chest X-ray 
images and by applying a proposed weighted as arranging method which gives importance to different sensitivities 
of DL models on different class types. 
\bibliographystyle{ieeetr} 
\bibliography{reference} 
\end{document}